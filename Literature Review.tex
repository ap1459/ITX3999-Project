\documentclass{article}
\usepackage{float}
\PassOptionsToPackage{hyphens}{url} %Should fix reference spacing issue?
\usepackage{hyperref}
\usepackage{graphicx}
\usepackage{titling}
\graphicspath{ {images/} }

%YEARS NOT SHOWING ON CITES FOR WEBSITE SECTION
\author{Andreas Pavlou}
\title{ITX3999 Literature Review \& First Iteration}
\date{03/01/2018}

\begin{document}
\maketitle
\newpage


\section{Introduction}

Education on online safety is taught through a variety of different mediums; websites offer hubs of information with interactive features and links to many other relevant resources. Mobile applications offer interactivity and the ability to use when internet connection is limited. Many organisations also press forward in creating resources, whether they be in the form of websites, apps, physical print, etc to help educate individuals on the importance of online safety. This project will be taking the form of an app, therefore this will be the primary area of interest.


\section{Existing applications}

'DigitalCitizen' is the first result in the iOS app store when using the search term 'online safety'. This app, from its iOS store description tells us that it is designed for middle to high school students, with educational material regarding how to stay safe online, the ethical use of the internet and how students can deal with cyberbullying. It makes use of videos, games and quizzes also. It is targeted and developed for a US-based audience, which becomes apparent from analysing their developers website. This app however can only be fully accessed with an authentication code from a school; Locking the app behind an authentication code means that it is denying access to people who may be interested in learning, even those who perhaps are currently in school, but their respective school does not utilise the app in their curriculum, hence no activation code. \cite{noauthor_digitalcitizen_nodate} \\

\begin{figure} [H]
\includegraphics[scale=0.3]{DigitalCitizen}
\centering
\caption{Digital Citizen on the iOS app store}
\end{figure} 

\indent 	Another app named 'Child Safety Handbook' is the second result on the iOS store using the term 'Online privacy'. This app requires a payment to be made in order to purchase individual copies of the safety handbooks, or for an annual subscription plan. This handbook, according to the iOS store description covers a lot more than just online safety, including but not limited to bullying, first aid, fire prevention and so on. Locking the materials behind a paywall will limit its intended audience and may deter individuals who wish to learn. The proposed app differs in that it will be free, and focus specifically on online safety. \cite{noauthor_child_nodate} \\ 

\begin{figure} [H]
\includegraphics[scale=0.3]{Handbook}
\centering
\caption{Child Safety Handbook on the iOS app store}
\end{figure} 

\indent	'Kids, Children \& Teens Internet Safety' is the fifth result on the iOS store search using the term 'online safety', and costs {\texttt{�}}2.99 to purchase. This app offers a range of material regarding how a young person can stay safe online, also featuring videos but no games or quizzes. Similar to the app previously discussed, since it is locked behind a payment it is perhaps off-putting to parents who could make use of a different and free resource, whether that is a different app, or a website. \cite{noauthor_kids_nodate} \\ 

\begin{figure} [H]
\includegraphics[scale=0.3]{InternetSafety}
\centering
\caption{'Kids, Children \& Teens Internet Safety' on the iOS app store}
\end{figure} 

\indent	Using the term 'online safety guide' in the Google Play store yields 'CyberSecurity Internet Safety' which appears to be a rather simplistic guide to online safety regarding a handful of subjects, notably online banking and online shopping. There does not appear to have been a great deal of effort gone into the user interface or overall aesthetics of the app. %NEED TO ADD REFERENCE HERE

\begin{figure} [H]
\includegraphics[scale=0.4]{CyberSecurity}
\centering
\caption{CyberSecurity Internet Safety' on the Google Play store}
\end{figure} 


\section{Existing websites}

Simply typing 'online safety' into Google, results in a plethora of resources, the first page of results having links to the NSPCC, BBC, and many other organisations websites. A focus seems to be applied to children online safety as opposed to general online safety, if one goes by the Google search results on the first page. \\
\indent	The UK Safer Internet Centre is a partnership of three organisations: Childnet Internatoinal, Internet Watch Foundation and SWGfL (South West Grid for Learning). This partnership as appointed by the European Commission and they have three core functions, which are - 

\begin{itemize}
	\item Awareness: To provide support and advice to young persons and children, as well as schools as parents.
	\item Helpine: This is to mainly to support professionals who work with young people regarding online safety issues
	\item  Hotline: This operates as an anonymous hotline where illegal content featuring minors can be reported 
\end{itemize}


Regarding their webpage section on young persons, they very clearly highlight key areas that pose a threat to online safety, such as inappropriate content, personal conduct of an individual (Sharing too much information online, etc), being harassed online and young persons being unaware of the hidden costs within games, apps, and so on. It also has a teacher/school staff section which features resources on how to plan lessons and curriculum, taking into account online safety, even including a section on how as a working professional you should behave online to ensure your professional reputation remains untarnished. \cite{centre_uk_nodate} \\ 

\begin{figure} [H]
\includegraphics[scale=0.3]{InternetCenter}
\centering
\caption{Screenshot of the organisations social media page}
\end{figure} 

\indent	Since the vast majority of web results in the Google search are targeted at helping children with online safety, the term 'online safety for adults' will be used in order to get some more mixed results. \\
\indent	On the first page of results is Get Safe Online, a website that offers practical advice and guidelines on how one can go about protecting themselves online, and it covers topics ranging from social networking, banking to online shopping. Their social networking sites section contains a straightforward bullet-point system highlighting the risks of social networks as another list detailing what can be done to better protect oneself. Judging by the advice given and the presentation of the website, it is mainly for an older audience and not so much children. \cite{GetSafeOnline} \\

\begin{figure} [H]
\includegraphics[scale=0.3]{GetSafeOnline}
\centering
\caption{GetSafeOnline's guide to social networking}
\end{figure} 

\indent	The ICO (Information Commissioners Office) also has a webpage dedicated to online safety, covering many topics, from protecting online identity to what actions one can take to better their online social media privacy. It covers topics the project will focus on which is social networking, informing the visitor on what they should do as well as their legal rights. The webpage is informative with external links to relevant resources, each section taking about 2-3 paragraphs of information. \cite{noauthor_online_nodate}\\

\begin{figure} [H]
\includegraphics[scale=0.35]{ICO}
\centering
\caption{A section of the ICO's guide to online safety}
\end{figure} 

\indent	The NSPCC (National Society for the Prevention of Cruelty to Children) website has their own online safety page, that has links to a telephone number that can be contacted to talk directly to someone to receive help, as well as online games that can be played to help children understand the topic in an entertaining format.  It also has learning resources for teachers to make use of, as well as an online course, which costs �30 and teaches about how to help kids stay safe online, covering topics from cyberbullying, extremism and so on. The NSPCC{\texttt{'}}s online resources regarding online safety are clearly aimed for helping children the most, therefore whilst they may offer practical and useful advice, it may no be as relevant to adults and not deal with certain topics such as social media and online banking, which children are too young to utilise. \cite{nspcc_online_nodate}\\
\indent	The BBC also have a resource on their website called Webwise, that is on the first page of the Google search. It offers ten tips regarding online safety, offering suitable advice such as guarding personal details and use pseudonyms.  This resource however was made in 2013 and could perhaps include more up-to-date information.\cite{noauthor_top_2013}\\
\indent	Another results on the first Google search page is from the Lewisham Councils website; It offers a few bullet points on the topics of security, social media, online shopping as well as for mobile devices. It also include links to report fraud to the police. It does not offer much information and the advice seems somewhat general and lacks any depth or guidance on how to actually go about implementing the proposed advice, an example being it states that complex passwords should be used, but has not stated what exactly makes a password complex. It even has an external link to the aforementioned Get Safe Online website. \cite{Lewisham}\\

\begin{figure} [H]
\includegraphics[scale=0.35]{Lewisham}
\centering
\caption{Lewisham council's simple online safety guide}
\end{figure} 

\indent	AgeUK, a charity dedicated to providing services and support to the over 60{\texttt{'}}s also has a webpage with tips regarding online safety. It offers good advice on ways in which one can protect themselves online, even suggesting appropriate anti-virus software, as well as PDF{\texttt{'}}s containing more information relating to internet security and avoiding scams. It also avoids using to much technical jargon and keeps the language at a level anyone can understand. \cite{AgeUK}\\














\section{Development technologies}

When it comes to constructing the proposed app, there are numerous options available, falling into the two key categories of native and hybrid. Both options will be explored and a conclusion come to as to which would be most suitable for this particular project. Since the proposed app will rely heavily on utilising an engaging, intuitive and straightforward user interface, this is the function that I will be focusing on and comparing across all available options of implementation. \\

\subsection{Native apps}

Apps that fall under this category are built only for a specific operating system, and are only available through the relevant app store, so if an app were created for iOS, it would need to be programmed with Objective-C or Swift in conjunction with the software Xcode, and would only be available via the iTunes App store \cite{stark_building_2010}. The benefit of Native apps is that they can make full use of the devices hardware unhindered.

\subsubsection{iOS}

Creating a native iOS app will require that Xcode, also known as Xcode Developer Tools is used, in conjunction with Objective-C, or the Swift programming language. Creating the basic user interface with Xcode is particularly straightforward. It has what is called the 'Main.storyboard' where we can add whatever we want in terms of buttons, text and other functionality and it is as simple as drag-and-dropping what functionality would like to where we would like it to appear on the screen. Xcode also offers a very useful simulator in which it is possible to view how the app will look like on any version of the iPhone or iPad, as well as allowing one to test the apps functionality, in terms of buttons, swiping the screen, rotation and so on. It also utilises an 'Auto Layout' feature which allows for specifying the constraints in regards to position and size for each view object we have on the screen, which is immensely useful given that one would expect the app to function properly on any iPhone/iPad \cite{keur_ios_2016}. It is also rather simple to create connections between two objects on screen, for instance if we wanted to make it that an action is taken when we push a particular button, it is easy to do so using connections. Once two objects are connected the revenant code can be added to implement the necessary actions.

\subsubsection{Android}

In order to produce a native Android app, the Java programming language will need to be used as well as Android Studio. Android Studio is particularly flexible as it can be downloaded for MacOS whereas its counterpart Xcode is not available to the Windows operating system. Developing an app for Android requires on average 40\% less code than iOS but requires 30\% more working hours, according to \cite{vilcek_comparative_2017}. Android Studio utilises a similar approach to Xcode in that it as simple as drag-and-dropping what functionality we want to the wherever we want it to appear on the mobile phone screen.

\subsection{Hybrid apps}

Hybrid app{\texttt{'}}s are exactly as the name would suggest, they are part native app as well as part web app. These apps are also able to access the hardware of the device as well as the benefit of being distribution through the platforms app store, so unlike native apps, it is not limited to just one store. These apps can be created using CPT{\texttt{'}}s (Cross Platform Tools), meaning the app can be developed using HTML, CSS and Javascript and will be wrapped inside a third part native app container \cite{mohamed_ali_wama16:_2016}. Example of such are PhoneGap and Ionic. 

\subsubsection{PhoneGap}

This framework does not offer the same functionality as the previously discussed Android and iOS frameworks in that there is no user interface to simply drag and drop elements, it does have the benefit though of using HTML, CSS and JavaScript, meaning that languages such as Objective-C and Java are not needed. The desktop application creates a server which an iOS device can connect which has the added benefit of being able to see how an app will look on a device in hand, as opposed to a desktop visualiser. In order to see changes it made, it requires the user to stop the server running and then start it again, though this can be temperamental and if it refuses to update, closing the app on the iPhone and running it again will fix it.

\subsubsection{Ionic}

Ionic is an open source mobile software development kit and allows for the creation of web and native mobile apps for each of the major app stores. Ionic just like PhoneGap does not utilise any form of drag-and-drop functionality like the previously discussed native development technologies. This framework is slightly more tricky as it requires one make more use of the command line, but similar to PhoneGap it requires that a server be created, and through this the files can be edited and the changes be shown in real-time on a chosen browser. It is also possible to make use of the Xcode and Android Studio plug-in visualisers, so it can be seen what the app would actually look like on a mobile phone.

\subsection{Native and Hybrid comparison}

Using either native or hybrid app development technology will not be without their advantages or disadvantages; all must be examined and a conclusion come to as to which method would best suit the proposed application.

\begin{itemize}
	\item \textbf{Platform limitations} - As previously discussed, developing a native app would greatly limit the potential user-base of the app, as it would be available to one and only one app store; if the proposed app were to be developed in iOS, it would mean that at a later date if we wanted this app to become available on the Android store, it would have be recoded using Android Studio, or it would simply have to be recoded using a hybrid framework. Therefore Hybrid frameworks have the clear advantage that the developed application will only needed to be coded once and will available to all platforms.
	\item \textbf{Interface development} - Using a native application developer allows for the use of many tools and widgets which allow for a straightforward design of an applications interface, allowing the user to simply drag-and-drop whatever element they desire to which part of the screen the would like it to appear, which is far less time consuming than coding from scratch.
	\item \textbf{Responsiveness \& performance} - Native apps usually provide for a far more fluid experience and general better performance than their hybrid counterparts; An example is how in native applications, clicks feel far more responsive and scrolling down a page is smooth, whereas in a hybrid application, a user may have to click more than once to get a response and scrolling might perceive a delay in the the frames loading in sequence.\cite{khandeparkar_introduction_2015}
	\item \textbf{Programming language} - Creating a native application requires that one is knowledgable in the respective native programming language. Android studio applications requires that the JAVA programming language is used and iOS requires either that Swift or Objectice-C is used. Hybrid apps on the other hand only require that HTML, CSS and Javascript are used, which are far simpler to implement than their native counterparts.
\end{itemize}

\section{Conclusion}

From the experience gained from utilising both native and hybrid development softwares, I believe PhoneGaps functionality felt most suitable for the proposed project.











\newpage
\section{First Iteration}


The proposed app will need to be able to deliver informative and practical information to users in regard to online safety concerning a multitude of sub-topics, such as social media, email, and so on. In that regard, identifying and understanding the user requirements is of paramount importance, in order for the project to be successful; A user requirement is a pivotal piece of input information that helps in the development of new products, or to simply improve product design, therefore a clear understanding and grasp on the user requirements is key to making reliable and optimal design and functionality decisions for a product \cite{wang_user_2017}. \\
\indent	In order to attain meaningful and useful information from individuals, three people will be asked, amongst many questions, what they believe the requirements should be and what they would expect from such an app, as opposed to asking perhaps hundreds of individuals or so, where due to time constraints, it would be both impractical and impossible to get any meaningful feedback.


\section{User Responses}
	
Having emailed several charities in regards to the proposed app, a couple replied and these charities focused mainly on the younger demographic of users such as primary to secondary school; the UK Safer Internet Centre thought that targeting an app to a younger demographic may be difficult as it is not viewed as particularly 'cool' amongst peers to use such an app. They even suggested that if such an app were to be utilised by school kids, it could even make them a target for bullying. Though they recognised the importance of the topic and suggested if a creative spin were to be put on such an educational app, it may find better success. Mind, a UK based charity focusing on helping those with mental health problems, responded stating that they found it a very worthwhile pursuit, but unfortunately were unable to offer any detailed critique, as they said they do not have the resources to help students with projects.

In order to gain the user requirements, discussion were had with three individuals and the responses gathered provided for a refreshing and interesting insight into the proposed app, in regard to how it should operate, what it needs to do and so on. Firstly, users were asked if they had experienced any form of cyberbullying or a hack, and 2 out of 3 had. It was then asked if they knew what to do in the event of a hack, and if not what resource would they use in order to find out what to do, and all stated they would utilise Google to find an appropriate answer, as well as they have no knowledge of what to do without searching online. \\
\indent	Users were asked if they made use of any form of security software or related software to the proposed app, and only one respondent used an anti-virus software and this user had a Microsoft operating system, whilst the other two did not use any kind of specialist software, and were using Mac computers and believed such software was unnecessary. \\
\indent	In order to better understand user experience with apps, it was asked what irritates users most when it comes to using an app. All expressed they disliked being made to sign up using their Facebook or email, and two respondents did not like having to create an account with an app and provide their email, as they did not appreciate the lengthy process and the idea of newsletter emails that could spam their inboxes. \\
\indent	An interesting point was brought in the discussions, where one respondent made note that whilst they would be more inclined to use a quick Google search to find an answer, they would be more interested in using an app such as the proposed app, if it also included the latest news in regard to online security, as well as news on the latest high-profile hacks, such as banks, companies, etc that could perhaps be relevant to them. 
	
\subsection{Look \& Feel}
	
Discussions were had on how the proposed app should look and how it should feel, and as a result it was learned that users appreciated apps that make use of a monochromatic, simple colour system, as opposed to multiple colours or a design that looks out-of-place, as this could result in the app looking unprofessional as well as lack general aesthetic. When users were questioned on how they believe the balance of on-screen information should be, they felt as if there should be a fine balance, where enough information should be on screen that is appropriate; therefore not too much, nor too little information must be on screen at any one time, but rather enough for them to understand. \\
	
To summarise, the key user requirements are 

\begin{itemize}
	\item The app should avoid any form of registration, whether via email or Facebook
	\item The app should include some form of news feed (RSS) with the latest tech news
	\item The app must have a suitable amount of information on screen regarding the topics, not too much but not too little
	\item The app should be easy to navigate and straightforward
	\item The app should have a professional and and high-quality look/aesthetic 
\end{itemize}
	
\section{Design \& Implementation}

Based upon the received feedback, a paper prototype will be created using the JUSTINMIND service, a high-fidelity prototyping tool for mobile and web apps. Utilising this software will allow for a high-quality visual representation of what the app will look like, as well as to help showcase the basic functionality, in terms of navigation through the app. The app has been developed with simplicity and usability in mind, as cluttering the page with too many options, images or such may detract from the experience and lead to frustration if the intended user cannot find what they are looking for, or perform the task they wish to carry out.
	
\begin{figure} [H]
\includegraphics[scale=0.5]{Quiz}
\centering
\caption{Example question from the social media quiz}
\end{figure} 

This screenshot from the quiz section of the app has been kept straightforward and simple, in terms of functionality they will be able to answer the question and then tap 'Next Question' which will take them to the subsequent question. As for the design, it has been kept rather simple, though the colour scheme could be subject to change throughout the second iteration.

\begin{figure} [H]
\includegraphics[scale=0.5]{SocialMedia}
\centering
\caption{First page of the social media topic}
\label{fig:SocialMedia}
\end{figure} 

Figure~\ref{fig:SocialMedia} is what the first page of the social media topic will look like. Subsequent pages will have further information in regards into how to better protect your identity on social media, as well as a brief introduction of each platform shown on the screenshot (Facebook, Twitter and Instagram). Again, the simplistic and straightforward layout and style is maintained, as so the pages will not be to cluttered and will be easy to navigate.

\begin{figure} [H]
\includegraphics[scale=0.5]{SocialMedia2}
\centering
\caption{Second social media page}
\label{fig:SocialMedia2}
\end{figure} 

Figure~\ref{fig:SocialMedia2} is the next social media page, where users will be allowed to select whichever social media they like and it shall then take them to correlating page where a brief history shall be given on the chosen platform, as well as security advice.

\section{Evaluation}

Having created a paper prototype, it would now be suitable and beneficial to review it with the aforementioned users and get feedback in order to assess if the prototype is what they expected, and whether any changes or improvements need to be made to the design itself or even the project scope. \\
\indent	The three previous individuals used to attain the user requirements were then called upon once more to evaluate the paper prototype, and they did this by viewing the app through JustInMind's 'View on device' feature which allowed for the app to be used on a supplied iPhone 5S as if it were a functioning and real app. Users were told that the app was only a prototype and were then asked to try out the app for themselves and navigate around, assess the available features and whether they believe them to be good or if they believed they require more work. \\

Users were asked in particular - 

\begin{itemize}
	\item If they felt the app was straightforward to navigate
	\item If they felt the features available were useful
	\item Do they think there are potential features missing
	\item What they thought about the aesthetics/look of the app
\end{itemize}


Users stated that they really liked the quiz section, they believed the layout and approach to be to the standard they would expect from such an app. Users also critiqued the style/aesthetics, and because of this the colour scheme was changed from the 'dull' grey and white to a light blue and white, which users felt was more professional. Users also stated that the RSS feed which would deliver the latest news regarding hacks/security should be on the main page, so that it would instantly be there as they open the app, as opposed to having to navigate to a single page solely for that. They also suggested a menu bar which would be situated on the bottom of each page, where they could switch between sections would make the app easier to navigate. \\
\indent	At this stage in the process, the paper prototype has met most of the earlier discussed user requirements, such as the straightforward navigation and colour scheme, as well as not utilising any form of user registration.































\bibliographystyle{apalike}
\bibliography{LitReview,FirstIteration}



\end{document}