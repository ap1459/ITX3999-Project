\documentclass{article}
\usepackage{float}
\PassOptionsToPackage{hyphens}{url} %Should fix reference spacing issue?
\usepackage{hyperref}
\usepackage{graphicx}
\usepackage{titling}
\graphicspath{ {images/} }

\author{Andreas Pavlou}
\title{ITX3999 Final Report}
\date{23/02/2018}

\begin{document}
\maketitle
\newpage

\tableofcontents
\newpage

\section{Abstract}

The following work is the culmination of an academic years worth of work attempting to build a hybrid mobile application on the topic of online safety. It focuses in on particular themes, such as safety in regard to the use of social media, online messengers, and so on. Originally it was to be targeted to a younger audience, though after discussion with a charity, it was deemed more suitable to an audience of secondary school age, and upwards. The application also features quizzes to test users learning, as well as an RSS feed to deliver to them the latest news regarding hacks and general IT security developments.


\section{Introduction}

Safety online has never been more important; from ever-emerging and ever-evolving threats, such as the sophisticated ransomware attacks on the NHS in May 2017, to cyberbullying which one in three teenagers have experienced \\
\cite{association_number_2014}, it is imperative that online safety is taught to ordinary people who may not be particularly well-versed in the best safety practices. Individuals between the ages of 18 to 29 make up 83\% of social media users and 43\% of this aget bracket believe that their personal data is less safe than compared with five years ago, a study  \cite{pew_research_2012} has shown. 



\newpage

\section{Existing applications}

'DigitalCitizen' is the first result in the iOS app store when using the search term 'online safety'. This app, from its iOS store description tells us that it is designed for middle to high school students, with educational material regarding how to stay safe online, the ethical use of the internet and how students can deal with cyberbullying. It makes use of videos, games and quizzes also. It is targeted and developed for a US-based audience, which becomes apparent from analysing their developers website. This app however can only be fully accessed with an authentication code from a school; Locking the app behind an authentication code means that it is denying access to people who may be interested in learning, even those who perhaps are currently in school, but their respective school does not utilise the app in their curriculum, hence no activation code. \cite{noauthor_digitalcitizen_nodate} \\

\begin{figure} [H]
\includegraphics[scale=0.3]{DigitalCitizen}
\centering
\caption{Digital Citizen on the iOS app store}
\end{figure} 

\indent 	Another app named 'Child Safety Handbook' is the second result on the iOS store using the term 'Online privacy'. This app requires a payment to be made in order to purchase individual copies of the safety handbooks, or for an annual subscription plan. This handbook, according to the iOS store description covers a lot more than just online safety, including but not limited to bullying, first aid, fire prevention and so on. Locking the materials behind a paywall will limit its intended audience and may deter individuals who wish to learn. The proposed app differs in that it will be free, and focus specifically on online safety. \cite{noauthor_child_nodate} \\ 

\begin{figure} [H]
\includegraphics[scale=0.3]{Handbook}
\centering
\caption{Child Safety Handbook on the iOS app store}
\end{figure} 

\indent	'Kids, Children \& Teens Internet Safety' is the fifth result on the iOS store search using the term 'online safety', and costs {\texttt{�}}2.99 to purchase. This app offers a range of material regarding how a young person can stay safe online, also featuring videos but no games or quizzes. Similar to the app previously discussed, since it is locked behind a payment it is perhaps off-putting to parents who could make use of a different and free resource, whether that is a different app, or a website. \cite{noauthor_kids_nodate} \\ 

\begin{figure} [H]
\includegraphics[scale=0.3]{InternetSafety}
\centering
\caption{'Kids, Children \& Teens Internet Safety' on the iOS app store}
\end{figure} 

\indent	Using the term 'online safety guide' in the Google Play store yields 'CyberSecurity Internet Safety' which appears to be a rather simplistic guide to online safety regarding a handful of subjects, notably online banking and online shopping. There does not appear to have been a great deal of effort gone into the user interface or overall aesthetics of the app. %NEED TO ADD REFERENCE HERE

\begin{figure} [H]
\includegraphics[scale=0.4]{CyberSecurity}
\centering
\caption{CyberSecurity Internet Safety' on the Google Play store}
\end{figure} 


\section{Existing websites}

Simply typing 'online safety' into Google, results in a plethora of resources, the first page of results having links to the NSPCC, BBC, and many other organisations websites. A focus seems to be applied to children online safety as opposed to general online safety, if one goes by the Google search results on the first page. \\
\indent	The UK Safer Internet Centre is a partnership of three organisations: Childnet Internatoinal, Internet Watch Foundation and SWGfL (South West Grid for Learning). This partnership as appointed by the European Commission and they have three core functions, which are - 

\begin{itemize}
	\item Awareness: To provide support and advice to young persons and children, as well as schools as parents.
	\item Helpine: This is to mainly to support professionals who work with young people regarding online safety issues
	\item  Hotline: This operates as an anonymous hotline where illegal content featuring minors can be reported 
\end{itemize}


Regarding their webpage section on young persons, they very clearly highlight key areas that pose a threat to online safety, such as inappropriate content, personal conduct of an individual (Sharing too much information online, etc), being harassed online and young persons being unaware of the hidden costs within games, apps, and so on. It also has a teacher/school staff section which features resources on how to plan lessons and curriculum, taking into account online safety, even including a section on how as a working professional you should behave online to ensure your professional reputation remains untarnished. \cite{centre_uk_nodate} \\ 

\begin{figure} [H]
\includegraphics[scale=0.3]{InternetCenter}
\centering
\caption{Screenshot of the organisations social media page}
\end{figure} 

\indent	Since the vast majority of web results in the Google search are targeted at helping children with online safety, the term 'online safety for adults' will be used in order to get some more mixed results. \\
\indent	On the first page of results is Get Safe Online, a website that offers practical advice and guidelines on how one can go about protecting themselves online, and it covers topics ranging from social networking, banking to online shopping. Their social networking sites section contains a straightforward bullet-point system highlighting the risks of social networks as another list detailing what can be done to better protect oneself. Judging by the advice given and the presentation of the website, it is mainly for an older audience and not so much children. \cite{GetSafeOnline} \\

\begin{figure} [H]
\includegraphics[scale=0.3]{GetSafeOnline}
\centering
\caption{GetSafeOnline's guide to social networking}
\end{figure} 

\indent	The ICO (Information Commissioners Office) also has a webpage dedicated to online safety, covering many topics, from protecting online identity to what actions one can take to better their online social media privacy. It covers topics the project will focus on which is social networking, informing the visitor on what they should do as well as their legal rights. The webpage is informative with external links to relevant resources, each section taking about two to three paragraphs of information. \cite{noauthor_online_nodate}\\

\begin{figure} [H]
\includegraphics[scale=0.35]{ICO}
\centering
\caption{A section of the ICO's guide to online safety}
\end{figure} 

\indent	The NSPCC (National Society for the Prevention of Cruelty to Children) website has their own online safety page, that has links to a telephone number that can be contacted to talk directly to someone to receive help, as well as online games that can be played to help children understand the topic in an entertaining format.  It also has learning resources for teachers to make use of, as well as an online course, which costs �30 and teaches about how to help kids stay safe online, covering topics from cyberbullying, extremism and so on. The NSPCC{\texttt{'}}s online resources regarding online safety are clearly aimed for helping children the most, therefore whilst they may offer practical and useful advice, it may no be as relevant to adults and not deal with certain topics such as social media and online banking, which children are too young to utilise. \cite{nspcc_online_nodate}\\
\indent	The BBC also have a resource on their website called Webwise, that is on the first page of the Google search. It offers ten tips regarding online safety, offering suitable advice such as guarding personal details and use pseudonyms.  This resource however was made in 2013 and could perhaps include more up-to-date information.\cite{noauthor_top_2013}\\
\indent	Another results on the first Google search page is from the Lewisham Councils website; It offers a few bullet points on the topics of security, social media, online shopping as well as for mobile devices. It also include links to report fraud to the police. It does not offer much information and the advice seems somewhat general and lacks any depth or guidance on how to actually go about implementing the proposed advice, an example being it states that complex passwords should be used, but has not stated what exactly makes a password complex. It even has an external link to the aforementioned Get Safe Online website. \cite{Lewisham}\\

\begin{figure} [H]
\includegraphics[scale=0.35]{Lewisham}
\centering
\caption{Lewisham council's simple online safety guide}
\end{figure} 

\indent	AgeUK, a charity dedicated to providing services and support to the over 60{\texttt{'}}s also has a webpage with tips regarding online safety. It offers good advice on ways in which one can protect themselves online, even suggesting appropriate anti-virus software, as well as PDF{\texttt{'}}s containing more information relating to internet security and avoiding scams. It also avoids using to much technical jargon and keeps the language at a level anyone can understand. \cite{AgeUK}\\














\section{Development technologies}

When it comes to constructing the proposed app, there are numerous options available, falling into the two key categories of native and hybrid. Both options will be explored and a conclusion come to as to which would be most suitable for this particular project. Since the proposed app will rely heavily on utilising an engaging, intuitive and straightforward user interface, this is the function that I will be focusing on and comparing across all available options of implementation. \\

\subsection{Native apps}

Apps that fall under this category are built only for a specific operating system, and are only available through the relevant app store, so if an app were created for iOS, it would need to be programmed with Objective-C or Swift in conjunction with the software Xcode, and would only be available via the iTunes App store \cite{stark_building_2010}. The benefit of Native apps is that they can make full use of the devices hardware unhindered.

\subsubsection{iOS}

Creating a native iOS app will require that Xcode, also known as Xcode Developer Tools is used, in conjunction with Objective-C, or the Swift programming language. Creating the basic user interface with Xcode is particularly straightforward. It has what is called the 'Main.storyboard' where we can add whatever we want in terms of buttons, text and other functionality and it is as simple as drag-and-dropping what functionality would like to where we would like it to appear on the screen. Xcode also offers a very useful simulator in which it is possible to view how the app will look like on any version of the iPhone or iPad, as well as allowing one to test the apps functionality, in terms of buttons, swiping the screen, rotation and so on. It also utilises an 'Auto Layout' feature which allows for specifying the constraints in regards to position and size for each view object we have on the screen, which is immensely useful given that one would expect the app to function properly on any iPhone/iPad \cite{keur_ios_2016}. It is also rather simple to create connections between two objects on screen, for instance if we wanted to make it that an action is taken when we push a particular button, it is easy to do so using connections. Once two objects are connected the revenant code can be added to implement the necessary actions.

\subsubsection{Android}

In order to produce a native Android app, the Java programming language will need to be used as well as Android Studio. Android Studio is particularly flexible as it can be downloaded for MacOS whereas its counterpart Xcode is not available to the Windows operating system. Developing an app for Android requires on average 40\% less code than iOS but requires 30\% more working hours, according to \cite{vilcek_comparative_2017}. Android Studio utilises a similar approach to Xcode in that it as simple as drag-and-dropping what functionality we want to the wherever we want it to appear on the mobile phone screen.

\subsection{Hybrid apps}

Hybrid app{\texttt{'}}s are exactly as the name would suggest, they are part native app as well as part web app. These apps are also able to access the hardware of the device as well as the benefit of being distribution through the platforms app store, so unlike native apps, it is not limited to just one store. These apps can be created using CPT{\texttt{'}}s (Cross Platform Tools), meaning the app can be developed using HTML, CSS and Javascript and will be wrapped inside a third part native app container \cite{mohamed_ali_wama16:_2016}. Example of such are PhoneGap and Ionic. 

\subsubsection{PhoneGap}

This framework does not offer the same functionality as the previously discussed Android and iOS frameworks in that there is no user interface to simply drag and drop elements, it does have the benefit though of using HTML, CSS and JavaScript, meaning that languages such as Objective-C and Java are not needed. The desktop application creates a server which an iOS device can connect which has the added benefit of being able to see how an app will look on a device in hand, as opposed to a desktop visualiser. In order to see changes it made, it requires the user to stop the server running and then start it again, though this can be temperamental and if it refuses to update, closing the app on the iPhone and running it again will fix it.

\subsubsection{Ionic}

Ionic is an open source mobile software development kit and allows for the creation of web and native mobile apps for each of the major app stores. Ionic just like PhoneGap does not utilise any form of drag-and-drop functionality like the previously discussed native development technologies. This framework is slightly more tricky as it requires one make more use of the command line, but similar to PhoneGap it requires that a server be created, and through this the files can be edited and the changes be shown in real-time on a chosen browser. It is also possible to make use of the Xcode and Android Studio plug-in visualisers, so it can be seen what the app would actually look like on a mobile phone.

\subsection{Native and Hybrid comparison}

Using either native or hybrid app development technology will not be without their advantages or disadvantages; all must be examined and a conclusion come to as to which method would best suit the proposed application.

\begin{itemize}
	\item \textbf{Platform limitations} - As previously discussed, developing a native app would greatly limit the potential user-base of the app, as it would be available to one and only one app store; if the proposed app were to be developed in iOS, it would mean that at a later date if we wanted this app to become available on the Android store, it would have be recoded using Android Studio, or it would simply have to be recoded using a hybrid framework. Therefore Hybrid frameworks have the clear advantage that the developed application will only needed to be coded once and will available to all platforms.
	\item \textbf{Interface development} - Using a native application developer allows for the use of many tools and widgets which allow for a straightforward design of an applications interface, allowing the user to simply drag-and-drop whatever element they desire to which part of the screen the would like it to appear, which is far less time consuming than coding from scratch.
	\item \textbf{Responsiveness \& performance} - Native apps usually provide for a far more fluid experience and general better performance than their hybrid counterparts; An example is how in native applications, clicks feel far more responsive and scrolling down a page is smooth, whereas in a hybrid application, a user may have to click more than once to get a response and scrolling might perceive a delay in the the frames loading in sequence.\cite{khandeparkar_introduction_2015}
	\item \textbf{Programming language} - Creating a native application requires that one is knowledgable in the respective native programming language. Android studio applications requires that the JAVA programming language is used and iOS requires either that Swift or Objectice-C is used. Hybrid apps on the other hand only require that HTML, CSS and Javascript are used, which are arguably far simpler to implement than their native counterparts.
\end{itemize}

\section{Literature Review Conclusion}

From the experience gained from utilising both native and hybrid development softwares, and comparing the features they have to offer, I believe that PhoneGaps functionality felt most suitable for developing the proposed project.






%% FIRST ITERATION ---------------------------------------------------------------------------------------------------------------------------------------------------------------------------------




\newpage
\section{First Iteration}


The proposed app will need to be able to deliver informative and practical information to users in regard to online safety concerning a multitude of sub-topics, such as social media, email, and so on. In that regard, identifying and understanding the user requirements is of paramount importance, in order for the project to be successful; A user requirement is a pivotal piece of input information that helps in the development of new products, or to simply improve product design, therefore a clear understanding and grasp on the user requirements is key to making reliable and optimal design and functionality decisions for a product \cite{wang_user_2017}. \\
\indent	In order to attain meaningful and useful information in regard to the proposed app, three individuals will be asked, amongst many questions, what they believe the requirements should be and what they would expect from such an app, as opposed to asking perhaps hundreds of individuals or so, where due to time constraints it would be both impractical and impossible to get any meaningful feedback. Getting the requirements from a smaller group of people means that it will allow for a more in-depth and focused discussion which would simply not be possible with a larger number of people due to the aforementioned constraints.


\section{User Responses}

The following section will discuss the responses received from both the charities emailed and the individuals that were talked to in order to obtain the user requirements.

\subsection{Charity Responses}
	
Having emailed several charities in regards to the proposed app, only a couple replied and these charities focused mainly on the younger demographic of users such as primary to secondary school; the UK Safer Internet Centre thought that targeting an app to a younger demographic may be difficult as it is not viewed as particularly 'cool' amongst peers to use such an app. They even suggested that if such an app were to be utilised by school kids, it could even make them a target for bullying. Though they recognised the importance of the topic and suggested if a creative spin were to be put on such an educational app, it may find better success. Mind, a UK based charity focusing on helping those with mental health problems, responded stating that they found it a very worthwhile pursuit, but unfortunately were unable to offer any detailed critique or advice, as they said they do not have the resources to help students with projects.

\subsection{User Requirements}

In order to gain the user requirements, discussions were had with three individuals and the responses gathered provided for a refreshing and interesting insight into other individuals thoughts on the proposed app, in regard to how it should operate, what it needs to do and so on. Firstly, users were asked if they had experienced any form of cyberbullying or a hack, and 2 out of 3 said that they had. It was then asked if they knew what to do in the event of a hack, and if not, what resource would they use in order to find out what appropriate steps to take; all the individuals stated they would utilise Google in order to find an appropriate answer and also stated they have no knowledge of what to do without finding the answers searching online. \\
\indent	Users were asked if they made use of any form of security software or related software to the proposed app, and only one respondent used an anti-virus software and this user had a Microsoft operating system, whilst the other two did not use any kind of specialist software, and were using Mac computers and believed such software was unnecessary. \\
\indent	In order to better understand user experience with apps, it was asked what irritates users most when it comes to using an app. All expressed they disliked being made to sign up using their Facebook or email, and two respondents did not like having to create an account with an app and provide their email, as they did not appreciate the lengthy process and the idea of numerous newsletters and emails that could be sent to their inboxes. \\
\indent	An interesting point was brought in the discussions, where one respondent made note that whilst they would be more inclined to use a quick Google search to find an answer, they would be more interested in using an app such as the proposed app, if it also included the latest news in regard to online security, as well as news on the latest high-profile hacks, such as banks, companies, etc that could perhaps be relevant to them. 
	
\subsection{Look \& Feel}
	
Discussions were had on how the proposed app should look and how it should feel, and as a result it was learned that users appreciated apps that make use of a monochromatic, simple colour system, as opposed to multiple colours or a design that looks out-of-place, as this could result in the app looking unprofessional as well as lack general aesthetic. When users were questioned on how they believe the balance of on-screen information should be, they felt as if there should be a fine balance, where enough information should be on screen that is appropriate; therefore not too much, nor too little information must be on screen at any one time, but rather enough for them to understand. \\
	
To summarise, the key user requirements are 

\begin{itemize}
	\item The app should avoid any form of registration, whether via email or Facebook
	\item The app should include some form of news feed (RSS) with the latest tech news
	\item The app must have a suitable amount of information on screen regarding the topics, not too much but not too little
	\item The app should be easy to navigate and straightforward
	\item The app should have a professional and and high-quality look/aesthetic 
\end{itemize}
	
\section{Design \& Implementation}

Based upon the received feedback, a paper prototype will be created using the JUSTINMIND service, which is a high-fidelity prototyping tool for mobile and web applications. Using this software will allow for a high-quality visual representation of what the app will look like, as well as to help showcase the basic functionality, in terms of navigation through the app. The app has been developed with simplicity and usability in mind, as cluttering the page with too many options, images or such may detract from the experience and lead to frustration if the intended user cannot find what they are looking for, or perform the task they wish to carry out. The screenshots provided are an indication as to what the working prototype will look like, though they may be subject to minor changes throughout the next stage of development process where a working prototype will be produced, using PhoneGap. \\
\indent	The following section will go over and describe a number of screenshots of the paper prototype, there is however a public link available where it can be accessed in full, clicking here - 
\href{https://www.justinmind.com/usernote/tests/28638570/31681772/31681774/index.html}{JustInMind Prototype}. Also available here - \\
 https://www.justinmind.com/usernote/tests/28638570/31681772/31681774/index.html
	
	
\begin{figure} [H]
\includegraphics[scale=0.5]{HomeScreen}
\centering
\caption{Example question from the social media quiz}
\label{fig:HomeScreen}
\end{figure} 
	
Figure~\ref{fig:HomeScreen} will be the introductory page when first opening the app. The name 'OSCAR' was chosen which stands for 'Online Security Commodity And Resources'. An acronym was chosen because it allowed for a short, snappy and interesting name to be given to the app, which would help it stand out and avoid a tedious and uninspiring title which may not necessarily peak the interest of any potential users. The logo was designed and chosen for its simplicity to help with instant recognition.


	
	
	
\begin{figure} [H]
\includegraphics[scale=0.5]{Quiz}
\centering
\caption{Example question from the social media quiz}
\end{figure} 

This screenshot from the quiz section of the app has been kept straightforward and simple, in terms of functionality they will be able to answer the question and then tap 'Next Question' which will take them to the subsequent question. As for the design, it has been kept rather simple, though the colour scheme could be subject to change throughout the second iteration.

\begin{figure} [H]
\includegraphics[scale=0.5]{SocialMedia}
\centering
\caption{First page of the social media topic}
\label{fig:SocialMedia}
\end{figure} 

Figure~\ref{fig:SocialMedia} is what the first page of the social media topic will look like. Subsequent pages will have further information in regards into how to better protect your identity on social media, as well as a brief introduction of each platform shown on the screenshot (Facebook, Twitter and Instagram). Again, the simplistic and straightforward layout and style is maintained, as so the pages will not be to cluttered and will be easy to navigate.

\begin{figure} [H]
\includegraphics[scale=0.5]{SocialMedia2}
\centering
\caption{Second social media page}
\label{fig:SocialMedia2}
\end{figure} 

Figure~\ref{fig:SocialMedia2} is the subsequent social media page, where users will be allowed to select whichever social media they like by selecting the corresponding logo and it shall then take them to the respective page where a brief history shall be given on the chosen platform, as well as the all important security advice. Logos were chosen as opposed to a rather dull text list, as it is more visual and allows the user to interact with the logos they may recognise and learn the ones they have not seen before.


\begin{figure} [H]
\includegraphics[scale=0.5]{News}
\centering
\caption{The latest news section}
\label{fig:News}
\end{figure} 

Figure~\ref{fig:News} shows us the proposed look of the page which will contain all the latest news regarding IT security, the latest on any recent database hacks, and so on. Users will be able to scroll down to view all the latest news. This style was chosen as it is straightforward, compact and allows the user to simply scroll through and read through all the relevant news stories.


\section{Evaluation}

Having created a paper prototype, it would now be suitable and beneficial to review it with the aforementioned users and get feedback in order to assess if the prototype is what they expected, and whether any changes or improvements need to be made to the design itself or even the project scope. \\
\indent	The three previous individuals used to attain the user requirements were then called upon once more to evaluate the paper prototype, and they did this by viewing the app through JustInMind's 'View on device' feature which allowed for the app to be used on a supplied iPhone 5S as if it were a functioning and real app. Users were told that the app was only a prototype and were then asked to try out the app for themselves and navigate around, assess the available features and whether they believe them to be good or if they believed they require more work. \\

Users were asked in particular - 

\begin{itemize}
	\item If they felt the app was straightforward to navigate
	\item If they felt the features available were useful
	\item Do they think there are potential features missing
	\item What they thought about the aesthetics/look of the app
\end{itemize}


The users stated that they liked the name of the app; they said that found it particularly cool and reminiscent of how various artificial intelligences from film and television are given human names, an example comparison was made of the AI 'JARVIS' from the 2008 film 'Iron Man'. Users stated that they really liked the quiz section, they believed the layout and approach to be to the standard they would expect from such an app, but also stated that they would appreciate being able to see the results in a more visual and graphical format instead of a simple number result. Users also critiqued the style/aesthetics, and because of this the original colour scheme which was a rather 'dull' grey and white, was then changed to a light blue and white, which users felt was more professional and far better to look at. Users also stated that the RSS feed which would deliver the latest news regarding hacks/security should be on the main page, so that it would instantly be there as they open the app, as opposed to having to navigate to a single page solely for that. They also suggested a menu bar which would be situated on the bottom of each page, where they could switch between sections would make the app easier to navigate. \\
\indent	At this stage in the process, the paper prototype has met most of the earlier discussed user requirements, such as the straightforward navigation and colour scheme, as well as not utilising any form of user registration. The feedback gathered from this evaluation stage will be taken into consideration when beginning the next stage of the project, which is the construction of the working prototype.


\newpage


%% SECOND ITERATION ---------------------------------------------------------------------------------------------------------------------------------------------------------------------------------

\section{Second iteration}

At this stage, a basic working-prototype of the application has been created. It has limited functionality, in terms of navigation and such, as the full functionality will not be implemented until the third and final iteration. It would now be beneficial to ask the same individuals who gave feedback before, to give their thoughts on the current state of the application, so it can be evaluated and if any changes or fixes are necessary, they can be made. The application has been made making use of PhoneGap and HTML, CSS and JavaScript.

\section{User Requirements}

At this stage in the second iteration, it would be suitable to take the evaluation from the previous iteration in order to identify what the user requirements for this stage are. The following are the requirements as specified in the previous evaluation - 

\begin{itemize}
	\item The results of the quiz should be graphical and easy to see
	\item The RSS feed should be on the main page when opening the app
	\item Some form of menu-bar for quick navigation
\end{itemize}

These will be taken into account when designing and implementing the application in the following section.

\section{Design \& Implementation}

The prototype application has been created utilising PhoneGap, and has been programmed in HTML, CSS and JavaScript. The screenshots shown have been taken whilst the application has been in use on an iPhone 5S.

The following section will now show and discuss screenshots of the application in its current state.


\begin{figure} [H]
\includegraphics[scale=0.3]{secondIterationHomePage1}
\centering
\caption{Home page with RSS feed}
\label{fig:secondIterationHomePage1}
\end{figure} 

\begin{figure} [H]
\includegraphics[scale=0.3]{secondIterationHomePage2}
\centering
\caption{Home page scrolled down to show more of the RSS feed}
\label{fig:secondIterationHomePage2}
\end{figure} 

The home page was designed to be simple and allow for the user to instantly see what the application had to offer, the three key things being information on topics, the quizzes and the news feed section. The RSS feed was added beneath so if the user wanted to, they could scroll down and browse through, and the main sections were kept at the top so they would be instantly available so the user could decide what they would like to do. \\

\begin{figure} [H]
\includegraphics[scale=0.3]{secondIterationTopics}
\centering
\caption{Topics main page}
\label{fig:secondIterationTopics}
\end{figure} 

As shown in Figure~\ref{fig:secondIterationTopics}, the topic section has stayed true to its paper prototype design, maintaining a simple yet effective and easily-followable approach, taking care to ensure the user follows the linear progression of this part of the application. \\

\begin{figure} [H]
\includegraphics[scale=0.3]{secondIterationTopics2}
\centering
\caption{Topics main page}
\label{fig:secondIterationTopics2}
\end{figure} 

Figure~\ref{fig:secondIterationTopics2} shows us the next social media page users arrive on after continuing from  Figure~\ref{fig:secondIterationTopics}. The icon list was kept as users expressed this to be more engaging that a dull list of text. Upon clicking on an icon, it will take the user to the corresponding social media topic page wherein it will discuss the social media.

\begin{figure} [H]
\includegraphics[scale=0.3]{slideMenu}
\centering
\caption{Topics main page}
\label{fig:slideMenu}
\end{figure} 

Figure~\ref{fig:slideMenu} shows us what the page looks like after the user clicks on the hamburger menu icon in the top left of screen. This icon was chosen for its instantly-recognisable look, to those that have used many mobile applications, and it has the functionality of being able to take the user back to the home page of the app, or to an About page which will feature some basic information regarding who made the application and why. \\

\begin{figure} [H]
\includegraphics[scale=0.3]{Quiz1}
\centering
\caption{Social media quiz}
\label{fig:Quiz1}
\end{figure} 

\begin{figure} [H]
\includegraphics[scale=0.3]{Quiz2}
\centering
\caption{Social media quiz}
\label{fig:Quiz2}
\end{figure} 

Figure~\ref{fig:Quiz1} and Figure~\ref{fig:Quiz2} show the quiz for social media, which has been kept similar in their straightforward layout just like the paper prototype design, although it was decided that to keep all question on one page instead of multiple pages. This also allowed for the user to look over all their answers at the same time, in order to make changes if they so wish, before submitting. At this moment there are only four questions, and upon evaluation from users, it will be decided if this is a suitable amount for the bitesize knowledge and testing this application offers, or if they believe there should be more questions.

\begin{figure} [H]
\includegraphics[scale=0.3]{Quiz3}
\centering
\caption{Social media quiz results}
\label{fig:Quiz3}
\end{figure} 

Figure~\ref{fig:Quiz3} shows us what the current state of the graphical results looks like. Using a Google API, a simple and concise pie chart has been made showing the users score. A pie chart was chosen over any other form of graph as it was felt that users would be able to instantly work out their score whilst other forms of graphs may be confusing.



\section{Evaluation}

The users were once again called upon to give their thoughts regarding the application at this stage. Users were given the opportunity to use the app via a supplied iPhone 5, and were asked if they felt the app had met, or at least partially met the requirements that were discussed and agreed upon in the first iteration, and these were -

\begin{itemize}
	\item Do they think the lack of needing to register/make an account is a good thing?
	\item What they thought about the RSS feed
	\item If they felt a suitable amount of information was on screen at any one time
	\item If they felt the app was simple to navigate
	\item If they liked the way the app looked
\end{itemize}

Users appreciated the fact they could jump right in and use the app without any form of lengthy sigh-up or registration process. Users liked the inclusion of an RSS feed, though they also stated seeing it now on the home page makes the page look somewhat cluttered, and suggested that perhaps a unique page just for the news feed would be more suitable.


 At this stage, the actual text-content of the application uses mostly dummy text, users were still asked if they felt if it seemed a suitable amount, if they liked the font and if it was easy to read, and users stated that it was to a good standard, although perhaps it could be split up more into multiple, easy-to-digest paragraphs. Users appreciated the simple back button in the bottom left corner, though proposed that some form of drop-down menu or or such could be utilised so the users did not have to go through multiple pages to get to a certain page. All three users stated that they liked the way the application looked at this stage, and felt it was similar to the paper prototype, albeit a few different colours here and there, as well as layout changes of the buttons. \\

Users were also given specific tasks to not only test how well the app performed, but also to see if these tasks were straightforward enough to carry out with the current state of the app and its current navigational functionality. Users were asked to - 

\begin{enumerate}
	\item Navigate to the RSS news feed
	\item Navigate to the quiz section and complete a quiz
	\item Find information on a specific topic (i.e. Facebook)
	\item Once at a specific point in the app, navigate to the home page without using the back button
	\item All of the above but with the phone in landscape mode
\end{enumerate}

For task 1, users were all able to find the relevant news section with ease, as it is on the first page when opening up the app. They also saw the option to navigate to the separate page where the RSS feed was also stored, and this was done in order to test if they preferred it on a separate page or not. When asked to complete task 2, users expressed that if the quiz had larger buttons, it would make it easier to operate, though it wasn't a huge issue. They also suggested that perhaps the ability to know which questions they answered incorrectly as opposed to just receiving their overall score would be useful, in order to help them improve.. Task 3 was a pretty straightforward task to complete, users appreciated the use of icons to select a specific social media platform, as they were instantly recognisable and were preferable to a lists of just text. When users were asked to complete task 4, all understood what the hamburger icon at the top left of the screen was used for, they clicked on that and selected 'Home', this therefore being rather straightforward for them to complete and not an alien task given they all use mobile applications regularly. Task 5 had some issues, as not all content aligned effectively in landscape and this therefore requires some more work. When users were asked if they would see themselves using this app in landscape or portrait, they all stated that portrait is how they would use the vast majority of their mobile applications, the notable exception being if they were to watch a video.


























\bibliographystyle{apalike}
\bibliography{LitReview,FirstIteration,References,IOSProgramming}



\end{document}