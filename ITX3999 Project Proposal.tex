\documentclass{article}
\usepackage{float}
\usepackage{hyperref}





\author{Andreas Pavlou}
\title{ITX3999 Project Proposal}
\date{November 1, 2017}

\begin{document}
\maketitle
\newpage



\section{Introduction}

Safety online has never been more important; from ever-emerging and ever-evolving threats, such as the sophisticated ransomware attacks on the NHS in May, to cyberbullying which one in three teenagers have experienced \\
\cite{association_number_2014}, it is imperative that online safety is taught to ordinary people who may not be particularly well-versed in the best safety practices. Individuals between the ages of 18 to 29 make up 83\% of social media users and 43\% of this age bracket believe that their personal data is less safe than compared with five years ago, a study  \cite{pew_research_2012} has shown. 



\section{Project Idea}

This project proposes that an iOS app with a target audience of 18-29 year olds is created, featuring information relating to how one can safely use services such as social media, email and instant messengers. The focus will be made on social media, as the aforementioned demographic is by far, the largest consumer. The app will also feature advice on how to deal with cyberbullying /harassment as 41\% of adults have experienced some form of harassment online \cite{duggan_online_2017}. The app will also feature a quiz, wherein users can partake in simple multiple-choice questions to help with their learning.

\section{Existing Products} 

Utilising the iOS store, conducting searches using terms such as 'security education' yields app that are similar in nature, such as 'CompTIA Security {\texttt{+}} Exam Prep 2017 Edition', though this particular app is fundamentally different from the proposed app as it offers only exam help for a specific qualification. Another app is 'Learn IT \& Cyber Security Free', this app however seems to go somewhat in-depth and could possibly be too difficult for an individual who does not know anything beyond basic computing. 'DigitalCitizen' is a similar app available on iOS though its target market appears to be primary to secondary age school children and it appears to be more suited for a US audience. Typing 'Teach privacy' into the app store search yields 'TechSafe --- Privacy' which is a similar app, but again it seems to have a target demographic of children from primary school to secondary school age. The difference between these and the proposed app is that the proposed app will have a target audience of 18-29 in mind wherein it can offer relevant learning materials suited to such a demographic, with an emphasis on the information being helpful to a person who may not be completely computer literate, or who may not know much about computing in general. \\
\indent According to \cite{street_mobile_2017}, 91\% of 18-34 year olds own a smartphone, therefore this project proposes that an app is built, instead of a standard webpage. Whilst one may only be a Google search away from finding information, an app however, can offer convenience, ease-of-use and portability which a website cannot, especially if one has limited access to the internet. An app will also allow for a more streamlined navigation of relevant information and learning resources, as opposed to an entire website where it could be off-putting if it is too difficult to find what one may be looking for. \cite{noauthor_talking_nodate} make the argument that \textit{'Young people are searching for the 'App' {\texttt{...}} that tells them just how to accomplish what they want to accomplish as quickly and efficiently as possible. If one app does not work, they look for another'}. Therefore, it is favourable that an app is pursued as opposed to any other medium in order to effectively target the young person demographic.

\section{Plan}

The project requires that 3 iterations are completed consisting of 4 key elements, which are requirements, design, implementation and evaluation. The project will commence with a paper prototype, where the app will effectively be storyboarded, detailing every page and what subsequent pages they link to and any other relevant features.
The second stage will require that a working prototype with minimal functions is produced. This will showcase the core functionality of the app and will be able to give a clear indication on what works with the app, and what requires improvement.
The third and final stage will culminate in a fully working prototype; this will be the final product for this project and will showcase all the skills learnt.
Below is a table displaying the milestones and deliverables.

\begin{figure}[H]
\centering
\end{figure}

\begin{table}[H]
	\centering
	\begin{tabular}{| l | l | l |}
\hline
Deadline & 	Milestones & 				Deliverables \\
\hline
Week 6   & 	Submit project proposal &		Project Proposal \\
Week 8 &		Finish literature review &		Literature review \\
Week 11 &	Complete first iteration &		Paper prototype \\
Week 12 &	Submit interim report &		Interim report \\
Week 16 &	Complete second iteration & 	Working prototype (minimal function) \\
Week 20 &	Complete third iteration &		Final prototype \\
Week 24 &	Submit artefact on CD & 		Artefact \\
Week 24 &	Submit final report &			Final report \\
TBA &  		Viva Voce &				Viva Voce \\
\hline

	\end{tabular}
\caption{Milestones and Deliverables}
\end{table}

\section{Relevant Materials}

These materials will be made use of throughout the project to enhance and improve the artefact.



\begin{itemize}
	\item Stark (2010) Building iPhone apps with HTML, CSS, and JavaScript [Online]. O{\texttt{'}}Reilly. Available from: https://www.dawsonera.com/abstract/9781449389000 [Accessed 20th October 2017]
	\item Firtman (2013) Programming the Mobile Web. [Online] O{\texttt{'}}Reilly. Available from: https://www.dawsonera.com/abstract/9781449335632 [Accessed 20th October 2017]
	\item Allan (2010) Learning iPhone programming [Online] O{\texttt{'}}Reilly. Available from: https://www.dawsonera.com/abstract/9781449389215 [Accessed 20th October 2017]
	\item Weyl (2013) Mobile HTML5 [Online] O{\texttt{'}}Reilly.\\
	 Available from: https://www.dawsonera.com/abstract/9781491948897 [Accessed 20th October 2017]
	\item Pilone et al (2013) Head First iPhone and iPad Development [Online] O{\texttt{'}}Reilly. Available from: https://www.dawsonera.com/abstract/9781491950098 [Accessed 20th October 2017]
	\item Rideout (2006) iPhone 3D Programming [Online] O{\texttt{'}}Reilly. Available from: https://www.dawsonera.com/abstract/9781449391287 [Accessed 20th October 2017]
\end{itemize}






\bibliographystyle{apalike}
\bibliography{References}





\end{document}
